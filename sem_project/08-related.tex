\section{Related Work}

There is surprisingly little work that has focused either on the
specific mechanisms of geo-filtering or methodologies
designed to defeat them.  Certainly, there are large,
organized efforts at enumerating instances of Internet
censorship~\cite{encore,filasto2012ooni,echo-censorship} and there is
considerable work that examines methods of bypassing
blocking~\cite{tor,censorship-sok,censorship-empirical-study}.  However,
geo-filtering at the server-side is far less well-studied.

Afroz et al.~\cite{afroz2018exploring} performed a large-scale
measurement study and found that geo-filtering was ubiquitous on the
Internet.  A large number of commercial VPN providers, including many
of those listed in Table~\ref{tbl:costs}, advertise their services as
a means of getting around geo-fences.  Khan et
al.~\cite{vpn-ecosystem-imc2018} and Weinberg et
al.~\cite{proxies-geolocation-lying} independently analyzed the VPN
ecosystem, with both sets of authors concluding that VPN providers
regularly misrepresent the location of their endpoints.
Interestingly, however, the misclassification of the endpoints'
locations due to erroneous geolocation is not by itself problematic
for users who wish to defeat geofences, so long as the fenced website
similarly misattributes the VPN endpoint's location.  Poese et
al.~measure the accuracy of geolocation services and find that errors
are fairly common~\cite{ip-geolocation-unreliable}.

Numerous efforts have attempted to map out and explore the performance
of the Internet's domain name system (cf.~studies by Callahan and
Allman~\cite{callahan2013modern} and Jung et al.~\cite{jung2002dns},
and measurement platforms such as the RIPE Atlas).  This paper also
examines DNS performance, but focuses on the costs of choosing remote
DNS resolvers.  Finally, DNSSEC obviates the benefits of \sdns
services by preventing the type of forged DNS resolutions on which
\sdns depends.  However, DNSSEC has seen slow adoption and even the
resolvers that support DNSSEC often fail to validate the authenticity
of DNS records~\cite{dnssec-ecosystem}, indicating that \sdns will
likely continue to function for at least the short-term.



%%% Local Variables:
%%% mode: latex
%%% TeX-master: "../main"
%%% End:
